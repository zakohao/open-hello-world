\documentclass[fleqn]{jarticle}
\usepackage{graphicx}
\usepackage{amsmath}
\setlength{\mathindent}{0pt}
\begin{document}
%%
本研究でこれまで行ったことをまとめる
\begin{enumerate}
    \item 実験用ジョブセットとして,製品種別ごとの加工機械の利用順番,利用時間の組み合わせデータの自動生成システムの作成
    \item 生成されたデータを用いた生産機械のスケジューリングを総当たり法で厳密解を求める最短生産スケジュール算出システムの作成
    \item 生産スケジュールの可視化システムの作成
    \item 遺伝的アルゴリズムをベースとする,緩和手法による許容解算出と解の改善システムの作成
    \item 生産機械に加え複数台AGVを導入した生産スケジュールの最適化システムの作成
    \end{enumerate}
    以上を行ったうえで,本研究では機械とAGV両方同時スケジューリングを求め、現有な仕事を最短時間で終了する許容解を導出し,その許容解のもとで,機械の遊休時間を活用する余剰製品を提案するシステムの開発に取り組む.
\section{モデルの記述}
\section{モデル仮定}
本モデルでは受注生産を行う工場の製造工程をモデル化する.
\begin{enumerate}
    \item 注文票はさまざまな注文主からくる。
    \item 各注文書には製品$S$の種別が1種類が記載されている。注文する製品種別が異なる製品は異なる注文書で発注される.
    \item 注文書ごとに価格、納期、個数がある。
    \item 工場は、出荷する出荷日(納期)が同じ注文書$O_{\textit{i}}$をまとめ、これを注文書セットとする。
    \item 製造する製品種別に応じて、使用する加工機械、加工時間とその加工機械の使用順が予め与えられる。
    \item 工場は注文書セットで同一出荷として受注した注文書を納期日までに出荷できるよう、加工機械とAGVの最適なスケジューリングを行う。
    \item スケジューリングの結果得られた注文書セットに含まれる受注した製品を定められた個数すべて出荷できる日を最速出荷日とする.また最速出荷日は早いほどよいとする.
    \item 機械は注文書に記載された製品種別で指定された注文個数をすべて加工が終了するまで,作業を中途で終えたり,途中で他の製造作業を行うような割り込みはできないものとする。
    \item 製品種別が異なるための機械のセットアップタイムは考慮しない。
    \item 機械は製品種別ごとに定まる順番の通りに機械を使用して加工作業を行わなければならない。
    \item 出荷が等しい注文書,すなわちある注文書セットがすべて出荷可能とればその注文書セットでの加工作業は終了となる。
    \end{enumerate}
\subsection{記号定義}
\begin{align}
    O_{i} &\text{: 第$i$番に工場が受領した注文書} \nonumber\\
    S_{ij} &\text{: 第$i$注文書に記載された注文製品$S$の種別番号$j$} \nonumber\\
    P_{ij} &\text{: 第$i$注文書で受注した製品$S$の種別番号$j$の価格} \nonumber\\
    Q_{ij}&\text{: 第$i$注文書で受注した製品$S$の種別番号$j$の注文個数}\nonumber
\end{align}
\subsection{実験用ジョブセット生成}
一般的に企業への注文書の具体例は外部には公開されない.そのため本研究では注文書セットを予め様々な確率的に生成する計算機実験用の自動生成システムを作成し,実験用の注文書セットを作成する.
本節ではこの自動生成の手続きとそれにより作成した注文書セットについて述べる.