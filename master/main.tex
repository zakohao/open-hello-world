\documentclass[a4paper,11pt,dvipdfmx]{jsreport}

\usepackage{graphicx}
\usepackage{amsmath}
\usepackage{enumitem}

\usepackage{cleveref}

\usepackage{todonotes}

\usepackage{setspace}

\usepackage{booktabs}


% cleveref の日本語化(章)
\crefname{chapter}{第\!\!章}{第\!\!章}
\Crefname{chapter}{第\!\!章}{第\!\!章}

% 表示形式:第〈番号〉章
\crefformat{chapter}{第#2#1#3章}
\Crefformat{chapter}{第#2#1#3章}

\usepackage[backend=biber,style=numeric-comp,sorting=none]{biblatex}

\addbibresource{references/Detection_of_lodging_using_aerial_imagery/Detection_of_lodging_using_aerial_imagery.bib}
\addbibresource{references/drone/drone.bib}
\addbibresource{references/Estimation_of_Vegetation_Index/Estimation_of_Vegetation_Index.bib}
\addbibresource{references/inasaku/inasaku.bib}
\addbibresource{references/U-Net/U-Net.bib}
\addbibresource{references/seminar/seminar.bib}


% --- ページレイアウト ---
\usepackage[top=2.5cm,bottom=3cm,left=2.5cm,right=2.5cm]{geometry}

% --- 欧文フォントまわり(警告除去) ---
\usepackage[T1]{fontenc}
\usepackage{lmodern}
\usepackage{textcomp}

% --- 目次デザイン ---
\usepackage{tocloft}
\renewcommand{\cftchapfont}{\bfseries}
\setlength{\cftchapnumwidth}{4em}
\setlength{\cftchapindent}{2em}
\setlength{\cftsecindent}{4em}
\setlength{\cftsubsecindent}{6em}

% --- fancyhdr ---
\usepackage{fancyhdr}
\setlength{\headheight}{37pt} % ← 推奨値 36.7577pt より少し大きく
\pagestyle{fancy}
\fancyhf{}
\cfoot{\thepage}
\rhead{\chapterTitle}
\renewcommand{\headrulewidth}{0.4pt}

% --- ページスタイル ---
\fancypagestyle{TocStyle}{
  \renewcommand{\headrulewidth}{0pt}
  \fancyhf{}
  \chead{\large \bfseries 普及型ドローンを利用した稲の倒伏モニタリング \\[2mm]
  ~酒米:山田錦の圃場を対象に~}
}

\fancypagestyle{ChapStyle}{
  \renewcommand{\headrulewidth}{0pt}
  \fancyhf{}
  \cfoot{\thepage}
}

\fancypagestyle{AcknStyle}{
  \renewcommand{\headrulewidth}{0pt}
  \fancyhf{}
  \cfoot{\thepage}
}

% --- 章タイトルをヘッダーへ反映 ---
\makeatletter
\newcommand{\chapterTitle}{}
\def\@makechapterhead#1{
  \vspace*{20pt}
  {\parindent \z@ \raggedright \normalfont
    \Huge\bfseries \ifnum\c@chapter>0 第\thechapter 章\space \fi #1\par\nobreak
    \vskip 30pt
    \xdef\chapterTitle{\ifnum\c@chapter>0 第\thechapter 章\space \fi #1}}}
\def\@makeschapterhead#1{%
  \vspace*{20pt}%
  {\parindent \z@ \raggedright \normalfont
    \Huge\bfseries #1\par\nobreak
    \vskip 30pt
    \xdef\chapterTitle{#1}}
}
\makeatother

% --- section / subsection 設定 ---
\makeatletter
\renewcommand{\chapter}{
  \if@openright\cleardoublepage\else\clearpage\fi
  \global\@topnum\z@
  \secdef\@chapter\@schapter}
\renewcommand{\section}{\@startsection{section}{1}{\z@}
  {3.5ex \@plus 1ex \@minus .2ex}
  {2.3ex \@plus.2ex}
  {\normalfont\Large\bfseries}}
\renewcommand{\subsection}{\@startsection{subsection}{2}{\z@}%
  {3.25ex\@plus 1ex \@minus .2ex}
  {1.5ex \@plus .2ex}
  {\normalfont\large\bfseries}}
\makeatother

\setcounter{tocdepth}{2}

% ===============================
\begin{document}
% ===============================

% --- 表紙 ---
\begin{titlepage}
  \vspace*{2cm}
  \begin{center}
    \Large 関西学院大学
  \end{center}
  \begin{center}
    \Large 2026年度\space\space 修士論文
  \end{center}
  \vspace{1cm}
  \begin{center}
    \Large \bfseries 普及型ドローンを利用した稲の倒伏モニタリング \\[5mm]
    ~酒米:山田錦の圃場を対象に~
  \end{center}
  \vspace{7cm}
  \begin{center}
    \large 総合政策研究科\space\space 総合政策専攻
  \end{center}
  \begin{center}
    \large 指導教員\space\space 山田\hspace{1em} 孝子\space\space 教授
  \end{center}
  \vspace{1cm}
  \begin{center}
    \large 学生番号\space\space 49024016
  \end{center}
  \begin{center}
    \large 星川\hspace{1em} 裕志
  \end{center}
\end{titlepage}

% --- 目次 ---
\newgeometry{top=4cm,bottom=3cm,left=2.5cm,right=2.5cm}
\tableofcontents
\setcounter{page}{0}
\thispagestyle{TocStyle}
\newgeometry{top=2.5cm,bottom=3cm,left=2.5cm,right=2.5cm}

% --- 本文 ---
\onehalfspacing

\include{chapters/chap1.tex}

\include{chapters/chap2.tex}

\include{chapters/chap3.tex}

\include{chapters/chap4.tex}

\include{chapters/chap5.tex}

\thispagestyle{ChapStyle}
\printbibliography[title={参考文献}, heading=bibintoc]


\chapter*{謝辞}
\addcontentsline{toc}{chapter}{謝辞}
\thispagestyle{AcknStyle}

本研究を遂行し,本論文を完成させるにあたり,多大なるご協力とご助言を賜りましたすべての皆様に,心より感謝申し上げます. 

本研究の対象である酒米「山田錦」の圃場を経営されている和田様には, ご多忙の中にもかかわらず,圃場での調査やドローン撮影に快くご協力いただきました. また,実際の栽培現場に基づく貴重なお話やご意見を頂戴し,本研究を現場に即した内容として進めることができました。ここに深く感謝申し上げます. 

また,農業分野に関する専門的な知見や助言を賜りました高原様には,本研究の内容を検討するうえで多くの示唆をいただきました. 実践的な視点からのご助言は,研究全体の理解を深めるうえで大きな支えとなりました. 心より御礼申し上げます. 

\end{document}