%\documentclass[twocolumn]{jarticle}
\documentclass{jarticle}

\usepackage{jsaiac}
\usepackage[dvipdfmx]{graphicx}
%\usepackage{amsmath}
\usepackage{amsmath,amssymb}
\allowdisplaybreaks[4] 

\begin{document}

% 封面
\begin{titlepage}
\centering
\vspace*{4cm}
{\LARGE 修士卒業論文 \par}
\vspace{2cm}
{\Huge 遺伝的アルゴリズムと深層強化学習に基づく \\
ジョブスケジュール最適化問題 \\
―PCB加工プロセスをモデルとして―\par}
\vspace{1cm}
{Job Scheduling Optimization Based on Genetic Algorithms and Deep Reinforcement Learning \\
―A Model of the PCB Manufacturing Process―\par}

\vspace{3cm}
{\Large
関西学院大学大学院 \\
総合政策研究科 \\
汪 永豪 \\
2026年1月
}
\end{titlepage}

\clearpage

\section*{\centering 要旨}
\addcontentsline{toc}{section}{要旨}
近年、市場グローバル化が進む中、製品に対する市場ニーズは多様化しており、顧客ごとに仕様が異なる多品種少量生産方式への対応が重要となっている。
本研究では、プリント基板 (PCB) 製造工場を対象に、同一加工機械を複数回使用する生産プロセスをモデル化した上で、スケジューリング最適化に関する検討を行った。
静的な環境においては、遺伝的アルゴリズム (GA) および深層強化学習に基づく Deep Q-Network(DQN)を用いて最適化を行い,両手法の性能比較を実施した。
さらに、機械故障等による環境変動が発生する動的な環境で高速にスケジュールを生成可能な DQN により, 適応性能の向上を図った。
これらから静的および動的な生産環境におけるスケジューリングを検討する。\\

キーワード:インダストリーエンジニアリング、オペレーション・リサーチ、ジョブスケジューリング最適化問題、遺伝的アルゴリズム、深層強化学習

\vspace{2em}

\section*{\centering Abstract}
\addcontentsline{toc}{section}{Abstract}
In recent years, as market globalization has advanced, market demands for products have become increasingly diversified, making it essential to respond to high-mix, low-volume production systems in which specifications differ for each customer.
This study focuses on a printed circuit board (PCB) manufacturing factory and models a production process in which identical processing machines are used multiple times. Based on this model, scheduling optimization is investigated.
Under a static production environment, this study applies a genetic algorithm (GA) and a Deep Q-Network (DQN) based on deep reinforcement learning to perform scheduling optimization, and conducts a comparative evaluation of the performance of these two methods. 
Furthermore, under a dynamic production environment where environmental changes such as machine failures occur, this study aims to improve adaptability by utilizing DQN, which can generate schedules rapidly in response to such changes.
Through these approaches, this study examines scheduling optimization for both static and dynamic production environments.\\
Keywords : Industry Engineering, Operations Research, Job Shop Scheduling Optimization Problem, Genetic Algorithm, Deep Reinforcement Learning

\clearpage

% 目次
%\section*{目次}
\addcontentsline{toc}{section}{目次}
\tableofcontents
%\begin{enumerate}
    %\item 要旨
    %\item はじめに
    %\item 先行研究
    %\item 生産管理問題のモデル
        %\subitem 問題設定 
        %\subitem 用語の定義
    %\item 実験用データ自動生成ルール
        %\subitem データ自動生成の考え方
        %\subitem 実際の生成手続き
        %\subitem データ生成結果
    %\item 遺伝的アルゴリズムによる解の改善
        %subitem 遺伝的アルゴリズムの考え方(フローチャート)
        %subitem PPSの考え方(フローチャート)
        %subitem 実験手順
        %subitem 実験結果
        %subitem 実験考察
    %\item 深層強化学習による解の改善
        %subitem 強化学習の考え方(フローチャート)
        %subitem 深層強化学習の考え方(フローチャート)
        %subitem DQNの採用理由
        %subitem 実験手順
        %subitem 実験結果
        %\subitem 実験考察
    %\item 動的な環境における性能評価
        %subitem 
    %\item 遺伝的アルゴリズムと深層強化学習の結果と考察
    %\item 今後の課題
    %\item 謝辞
    %\item 参考文献
%\end{enumerate}
\clearpage

\title{
\jtitle{遺伝的アルゴリズムと深層強化学習に基づく\\ジョブスケジュール最適化問題\\―PCB加工プロセスをモデルとして―}
\etitle{Job Scheduling Optimization Based on Genetic Algorithms and Deep Reinforcement Learning \\―A Model of the PCB Manufacturing Process―}
}

\jaddress{関西学院大学総合政策研究科, 兵庫県三田市学園上ヶ原, ili53007@kwansei.ac.jp}

\author{
   \jname{汪 永豪\first{}}
   \ename{Wang Yonghao}
\and
   \jname{山田 孝子\second{}}
   \ename{Takako Yamada}
}

\affiliate{
\jname{\first{}関西学院大学大学院総合政策研究科}%
\ename{Graduate School of Policy Studies, Kwansei Gakuin University}%
\and
\jname{\second{}関西学院大学総合政策学部メディア情報学科}
\ename{Department of Applied Informatics, School of Policy Studies, Kwansei Gakuin University}
}

\def\Style{``jsaiac.sty''}
\def\BibTeX{{\rm B\kern-.05em{\sc i\kern-.025em b}\kern-.08em%
 T\kern-.1667em\lower.7ex\hbox{E}\kern-.125emX}}
\def\JBibTeX{\leavevmode\lower .6ex\hbox{J}\kern-0.15em\BibTeX}
\def\LaTeXe{\LaTeX\kern.15em2$_{\textstyle\varepsilon}$}
\maketitle

\section{はじめに}

世界経済のグローバル化に伴い、製造業企業は他社より安い価格で注文を受ける必要がある.
いかにして低コストな生産システムと高品質な品質管理を両立させ,安定的に維持するかは、製造業が必ず直面すると言って過言ではない難しい問題である.
昨今ではこの両立にさらに安全管理を含むワークライフバランスを考えた労務管理が加わり,生産現場では達成するべき課題が複雑化している.
コスト削減のため、多くな企業は人間作業員の代わりに、製造工程のロボットを採用し,工場内物流には複積載搬送ロボット(AGV)の導入をすすめている.
AGVの導入により生産機械間の半製品や仕掛品の運搬は自動化された.
生産計画には,スケジューリングだけではなく,AGVの搬送計画も含まれ,経路や運搬すべき半製品の決定までを生産計画に含むことになった.
本研究では,以下のシステムを順次作成し,生産管理計画システムとして実装し,最適解を求める計算機実験を実施,システムの性能を検証した.

本研究でこれまで行ったことをまとめる
\begin{enumerate}
    \item 実験用ジョブセットとして,製品種別ごとの加工機械の利用順番,利用時間の組み合わせデータの自動生成システムの作成
    \item 生成されたデータを用いた生産機械のスケジューリングを総当たり法で厳密解を求める最短生産スケジュール算出システムの作成
    \item 生産スケジュールの可視化システムの作成
    \item 遺伝的アルゴリズムをベースとする,緩和手法による許容解算出と解の改善システムの作成
    \item 生産機械に加え複数台AGVを導入した生産スケジュールの最適化システムの作成
    \item 求められた生産スケジュールに基づく生産機械の遊休時間活用する余剰生産品提案システムの作成
    \end{enumerate}
    以上を行ったうえで、本研究では機械とAGV両方同時スケジューリングを求め、現有な仕事を最短時間で終了する許容解を導出し,その許容解のもとで,機械の遊休時間を活用する余剰製品を提案するシステムの開発に取り組む.
    このような問題は、JSSP(Job Shop Scheduling Problem)問題と言います。

\section{先行研究}
グローバル経済の進展や企業のコスト削減要求に伴い、生産現場では柔軟で効率的な生産体制の構築が求められている。
その中で、需要の多様化に対応するために、少量多品種生産が一般的となり、このような生産形態に適したスケジューリングの最適化が重要な課題となっている。
特に、ジョブショップスケジューリング問題(Job Shop Scheduling Problem: JSSP)は、生産システムにおける代表的な最適化課題として広く研究されている。
JSSPは、複数のジョブを限られた機械に割り当て、全体の処理時間や納期遅延を最小化することを目的とする問題である。
また、各ジョブには処理順序が定められており、この順序を厳密に守る必要がある点が特徴である。
この問題はNP困難であり、最適解を求めることは計算量的に非常に困難である。
そのため、これまで多くの研究では、遺伝的アルゴリズム(Genetic Algorithm: GA)や焼きなまし法(Simulated Annealing: SA)などのヒューリスティック手法を用いて、効率的に近似解を求めるアプローチが提案されてきた。
しかし、実際の製造現場では、機械の故障や新規ジョブの追加、処理時間の変動など、動的な要素が頻繁に発生する。そのような動的環境下では、あらかじめ定められたルールに基づくヒューリスティック手法では十分に対応できない場合がある。この問題を解決するため、近年では強化学習(Reinforcement Learning: RL)を用いて、環境変化に応じて自律的にスケジューリング方策を学習する手法が注目されている。
静的な環境においてはGAやSAなどの伝統的アルゴリズムが一定の成果を上げている一方で、動的な生産環境では強化学習に基づくアプローチのほうが、より高い適応性と柔軟性を示すことが報告されている。このように、JSSPにおけるスケジューリング最適化の研究は、従来の探索型手法から学習型手法へと発展しつつある。

実際に、近年の研究ではこの流れを反映した多様なアプローチが報告されている。
K.T. Chungら\cite{ref1}は現在工業5.0の時代に向けて、機械学習、特に強化学習を用いて、CO(combinatorial optimization)問題を解決しようについて、現在の研究進展を述べた。
M. Xuら\cite{ref2}は遺伝的プログラミングと強化学習を統合的に整理・比較しはJSSPスケジュール最適化問題ににおいて果たす役割と近年の研究動向を明らかにした。
C. Ngwuら\cite{ref3}は動的なJSSP問題は強化学習や進化的ヒューリスティック、機械学習を用いた手法でリアルタイムな意思決定は可能であることを示しました。
X.R. Shaoら\cite{ref4}は小規模かつ短時間のJSSP問題に対して、多層畳み込みニューラルネットワーク(ML-CNN)と反復局所探索(ILS)を組み合わせ、ML-CNNで全体経路を学習し、ILSで最適な局所経路を探索する手法が提案した。
L.B. Wangら\cite{ref5}は機械故障や再作業などの動的な環境でのJSSP問題につい深層強化学習(DRL)を用いた動的スケジューリング手法が提案され、PPOにより最適な方策を学習することで、リアルタイムな生産スケジューリングが可能であることが示された。
S. Leeら\cite{ref6}は射出成形金型製造プロセスにおいて、Deep Q-Networkを活用することで従来より総加重遅延を低減できる金型生産スケジューリングを最適化した。
C. Pickardtら\cite{ref7}は半導体製造プロセスにおいて、遺伝的プログラミング(GP)と確率的離散事象シミュレーションを組み合わせることで自動的に従来のルールより性能優れたなルールを生成した。
村山ら\cite{ref8}はJSSP問題に基づき、加工機械と複数積載型AGVの同時スケジューリングを対象とした遺伝的アルゴリズム(GA)手法が提案した。

これらの研究から、JSSPおよびその拡張問題に対して、静的環境ではヒューリスティック手法、動的環境では強化学習や深層学習手法が有効であることが確認されている。

\section{生産管理問題のモデル}
本研究は受注生産を行う工場の製造工程をモデル化する。
受注型ジョブショップスケジュール問題(Job Shop Scheduler problem:JSSP)に関する最適化問題であることで、事前にクライアント側からもらった注文書は不可欠なデータだ。

本研究は、実際のプリント基板(PCB)製造工場をモデルとした研究であり、本来であれば使用する注文書データは実生産に基づくものであることが望ましい。
しかしながら、PCB製造に関する実データは、企業の営業機密や顧客情報に直結するため、企業との共同研究等がない限り、その入手は極めて困難である。
そこで本研究では、先行研究に記載されているPCB製造工程を参考にし、各加工機械に対して加工役割を割り当てることで、PCB製造プロセスをモデル化する。
また、PCBの種類によって必要とされる加工時間、加工工程数、あるいは加工順序が異なることを考慮し、これらの差異をモデル内に反映させている。
本研究ではPCBを製品として定義する。例えば製品1はPCBの種類1、製品2はPCB板の種類2を表すものとする。
まとめて注文書をセットし、ジョブセットと呼ばれる。JSSPのスケジュール最適化対象と見られる。
なお、各機械の機能割り当ておよび加工製品に関する工程設定の詳細については、次章において述べる。

これから制限条件に書かれている符号の定義とモデル設定を紹介する。
\subsection{符号定義}
\begin{align*}
    O &\text{: 工場が受領した注文書の集合} \nonumber\\
    O_{i} &\text{: 第$i$番に工場が受領した注文書} \nonumber\\
    J_{s} &\text{: 製品$J$の種別番号$s$} \nonumber\\
    O_{i}J_{s} &\text{: 第$i$注文書に記載されたジョブ$J$の種別番号$s$} \nonumber\\
    lotJ_{i,s} &\text{: 第$i$注文書に記載された第$s$種別番号製品の注文ロット数} \nonumber\\
    n_{s} &\text{: 対象ジョブセットに記載された第$s$種別番号製品の合計注文個数} \nonumber\\
    n_{s} &= 10\sum_{i=1}^{\infty} lotJ_{i,s} \quad (\text{個数} = \text{ロット数} \times 10) \nonumber\\
    J_{s}F_{q} &\text{: 第$s$種別製品の第$q$番目の加工工程} \nonumber\\
    m_{s,q} &\text{: 第$s$種別製品の第$q$番目の加工工程に割り当てられる機械番号} \nonumber\\
    M &\text{: 使用可能な加工機械の集合} \nonumber\\
    M_{m} &\text{: 第$m$番加工機械} \nonumber\\
    t_{s,q} &\text{: 第$s$種別製品の第$q$番目の加工工程にかかる加工時間} \nonumber\\
    ft_{s,q} &\text{: 第$s$種別製品の第$q$加工工程の完了時刻} \nonumber\\
    ft_{m_{s,q}} &\text{: 機械$m_{s,q}$上で直前に処理された別製品の工程の完了時刻} \nonumber\\
    TJ_{s} &\text{: 第$s$種別製品の加工終了時間} \nonumber\\
    \text{makespan} &\text{: 全製品種別における最終工程完了時刻の最大値} \nonumber\\
    OM &\text{: 対象ジョブセット各加工工程機械割り当ての行列} \nonumber\\
    m_{s,q} &= OM_{s,q} \nonumber\\
    OT &\text{: 対象ジョブセット各加工工程にかかる加工時間の行列} \nonumber\\
    t_{s,q} &= OT_{s,q} \nonumber\\
    C &\text{: 制約条件} \nonumber\\
\end{align*}

\subsection{モデル設定}
\begin{enumerate}
    \item 注文票はさまざまな注文主からくる。
    \item 各注文書には製品$J$の種別が1種類以上が記載されている。
    \item 注文書ごとに製品ごとロット数が記載されている。
    \item 工場は、注文書$O_{i}$をまとめ、これをジョブセットとする。
    \item 製造する製品種別に応じて、使用する加工機械、加工時間とその加工機械の使用順が予め与えられる。
    \item 工場はジョブセットで同一出荷として受注した注文書を加工生産時間を最短にし、加工機械の最適なスケジューリングを行う。
    \item 機械は注文書に記載された製品種別で指定された注文個数をすべて加工が終了するまで,作業を中途で終えたり,途中で他の製造作業を行うような割り込みはできないものとする。
    \item 製品種別が異なるための機械のセットアップタイムは考慮しない。
    \item 製品の加工間の搬送時間はこのモデルでは考慮しない。
    \item 一つの機械は一つの加工工程のみ作業する。
    \item 機械は製品種別ごとに定まる順番の通りに機械を使用して加工作業を行わなければならない。
    \item 製品ごとに全部の加工工程が終了したら、その製品の加工スケジューリングが終了する。
    \item 全部の製品の加工スケジューリングが終了したら、ジョブセットのスケジューリングが終了する
    \end{enumerate}

\subsection{問題設定}
本研究は、JSSPを制約付き最適化問題として定式化し、目標関数としては:\\
\begin{equation}
\mathrm{opt.}\ JSSP = \min \left\{ \max_{s} TJ_{s} \right\}
\label{eq:objective}
\end{equation}
ここで,$TJ_s$ は第 $s$ 種別製品の最終加工工程の完了時刻を表す。
したがって,本最適化問題は,全ての製品種別における最終完了時刻の最大値,すなわち$makespan$を最小化することを目的とする。\\

制約条件としては:
\begin{align}
\text{s.t.}\quad \text{C1: }\ 
& ft_{s,q} \ge ft_{s,q-1} + t_{s,q}
\label{cond:C1} \\[6pt]
\text{s.t.}\quad \text{C2: }\ 
& ft_{s,q} \ge ft_{s',q'} + t_{s,q}
\label{cond:C2}
\end{align}


制約条件 $C1$ は、同一製品種別$J_s$における隣接する加工工程$F_{q-1}$と$F_q$の
加工順序制約を表しており、前工程が完了した後でなければ次工程を開始できないことを意味する。

制約条件 C2 における $ft_{s',q'}$ は、
加工工程 $J_sF_q$ と同じ機械 $m_{s,q}$ を使用するが,
製品種別が異なる別製品 $J_{s'}$ の加工工程 $J_{s'}F_{q'}$ の完了時刻を表す。
すなわち,$ft_{s',q'}$ は,同一機械上で $J_sF_q$ の直前に処理された工程の完了時刻に対応する。

制約条件 $C2$ は、各加工機械は同時刻に一つの加工工程しか処理できないという
機械リソース制約を表しており,ある工程が機械$M_m$上で処理中の場合,
次の工程はその加工工程が終了するまで開始できない。

JSSPにおいては、加工工程と加工機械の対応関係を表す機械割当行列$OM$と,
各加工工程に要する加工時間を表す加工時間行列$OT$が既知である。
$OM$と$OT$に関することは次章に述べる。

各製品種別の加工順序が与えられた場合,
各加工工程$J_sF_q$の完了時刻は次の式に基づいて算出することができる。

\begin{equation}
ft_{s,q} = \max \left( ft_{s,q-1},\,ft_{m_{s,q}} \right) + t_{s,q}
\label{eq:ft_calc}
\end{equation}

ここで,$ft_{s,q}$は第$s$種別製品の第$q$加工工程の完了時刻を表す。
本式は,各加工工程の完了時刻が,
同一製品における直前工程の完了時刻$ft_{s,q-1}$と,
同一加工機械$m_{s,q}$上で直前に処理された別製品工程の完了時刻$ft_{m_{s,q}}$
のいずれか遅い方に,加工時間$t_{s,q}$を加えた値として決定されることを示している。

ジョブショップスケジューリングの目的は、各加工機械をできるだけ連続的に稼働させるとともに、
各製品の加工工程を待ち時間なく加工することにより、総加工時間を最小化することである。
したがって、JSSPは待機している加工機械に対して加工対象となるジョブを割り当てる逐次意思決定問題として見られる。
スケジューリング過程は、加工機械上で各加工工程の処理完了イベントによって駆動される。
同一製品内における加工工程の順序制約および、各加工工程が指定された加工機械で加工されなければならないという制約を考慮すると、
待機中の加工機械に製品の加工工程を割り当てる際には、現時点で処理可能な加工工程が存在するかどうかを判断する必要がある。
処理可能な加工工程の集合が複数存在する場合には、その中から一つを選択して加工を開始する。
一方,現時点で処理可能な加工工程が存在しない場合には,次の加工工程完了イベントが発生するまで待機する。
この意思決定過程は次の図\ref{fig:JSSP flow chat}に示します。

\begin{figure}[htbp]
  \centering
  \includegraphics[width=0.5\linewidth]{figures/jssp_flow_chat.png}
  \caption{ジョブスケジューリング問題フローチャート}
  \label{fig:JSSP flow chat}
\end{figure}


\section{実験用データ自動生成ルール}
本研究では、実際のPCB製造工場をモデルとしたスケジューリング問題を対象とする。
本来であれば、行列 $OM$ および $OT$ に用いる機械加工順序データや加工時間データは、実工場から取得した実績データに基づくことが望ましい。
しかしながら、個々の製品の加工順序や加工時間、注文書などの生産情報は企業の営業機密および顧客情報に関連しており、
企業と共同研究等がない限り、入手することは困難である。
そこで本研究では、文献\cite{ref22}に記載された多層PCBの製造工程を基づく、各製品種別ごとの加工工程と使用機械を仮定、
その上で注文書のばらつきやロット構成を確率モデルにより生成することで、実工場を模擬した実験用データを自動生成する。


\subsection{データ自動生成の考え方}
多層PCBの製造工程は以下の加工工程は不可欠だ。

\begin{enumerate}
    \item 材料準備
    \item 材料表面の汚れや油分を除去し
    \item パターン転写(一回にフォトリソグラフィ方式とダイレクト印刷方式どちらを選ぶ)
    \item エッチングとレジスト除去(複数回加工する場合ある)
    \item 積層(複数回加工する場合ある)
    \item ドリル加工
    \item スルーホールめっき
    \item ソルダーレジスト塗布
    \item 表面処理
    \item 電気検査
    \item 包装
    \end{enumerate}

ここまでの加工ができましたら、出荷までの作業は終了とする。このような加工工程があって、加工機械に役割を割り当ては以下のように設定する。

\begin{align*}
    M_{1} &\text{: 銅箔付きの絶縁板1を取る} \nonumber\\
    M_{2} &\text{: 銅箔付きの絶縁板2を取る} \nonumber\\
    M_{3} &\text{: 表面の汚れや油分を除去し} \nonumber\\
    M_{4} &\text{: パターン転写(フォトリソグラフィ方式)} \nonumber\\
    M_{5} &\text{: パターン転写(ダイレクト印刷方式)} \nonumber\\
    M_{6} &\text{:エッチングとレジスト除去} \nonumber\\
    M_{7} &\text{:積層} \nonumber\\
    M_{8} &\text{:ドリル加工} \nonumber\\
    M_{9} &\text{:スルーホールめっき} \nonumber\\
    M_{10} &\text{:ソルダーレジスト塗布} \nonumber\\
    M_{11} &\text{:表面処理} \nonumber\\
    M_{12} &\text{:電気検査} \nonumber\\
    M_{13} &\text{:包装} \nonumber\\
\end{align*}

本研究は6種類の製品が生産可能な工場と仮定する。この6種類の製品の加工工程順序は以下のように設定する。

\[
OM =
\begin{bmatrix}
1 & 3 & 4 & 6 & 7 & 8 & 9 & 10 & 11 & 12 & 13 & 0 & 0 & 0 & 0 & 0 & 0 \\
2 & 3 & 5 & 6 & 7 & 8 & 9 & 10 & 11 & 12 & 13 & 0 & 0 & 0 & 0 & 0 & 0 \\
1 & 3 & 4 & 6 & 5 & 6 & 7 & 8 & 9 & 10 & 11 & 12 & 13 & 0 & 0 & 0 & 0 \\
2 & 3 & 5 & 6 & 4 & 6 & 7 & 8 & 9 & 10 & 11 & 12 & 13 & 0 & 0 & 0 & 0 \\
1 & 3 & 5 & 6 & 4 & 6 & 7 & 8 & 9 & 4 & 6 & 10 & 11 & 12 & 13 & 0 & 0 \\
2 & 3 & 5 & 6 & 4 & 6 & 7 & 8 & 9 & 4 & 6 & 5 & 6 & 10 & 11 & 12 & 13
\end{bmatrix}
\]


このOM行列はジョブセット各加工工程機械割り当ての行列、
$OM_{s,q} = \{1,2,\dots,12,13\}$ ということは製品$s$の第$q$番目加工工程の加工機械は$OM_{s,q}$である。
$OM_{s,q} = 0$の場合、製品$s$の第$q$番目加工工程自体はないことだ。

エッチングとレジスト除去、積層、スルーホールめっき、この三つの加工操作はPCBをまとめて操作可能な加工工程のことで、加工時間は定値に設定する。
この三つの加工工程を除いて、他すべての加工工程の加工時間は$n_{s} \times {単位時間}$の積だ。
次は各機械の単位加工時間もしくは定値加工時間を設定する。
\begin{align*}
    MT_{1} &\text{= 0.5(単位加工時間)} \nonumber\\
    MT_{2} &\text{= 0.5(単位加工時間)} \nonumber\\
    MT_{3} &\text{= 2(単位加工時間)} \nonumber\\
    MT_{4} &\text{= 8(単位加工時間)} \nonumber\\
    MT_{5} &\text{= 7(単位加工時間)} \nonumber\\
    MT_{6} &\text{= 250(基準加工時間)} \nonumber\\
    MT_{7} &\text{= 600(基準加工時間)} \nonumber\\
    MT_{8} &\text{= 3(単位加工時間)} \nonumber\\
    MT_{9} &\text{= 750(基準加工時間)} \nonumber\\
    MT_{10} &\text{= 5(単位加工時間)} \nonumber\\
    MT_{11} &\text{= 4(単位加工時間)} \nonumber\\
    MT_{12} &\text{= 3(単位加工時間)} \nonumber\\
    MT_{13} &\text{= 1(単位加工時間)} \nonumber\\
\end{align*}

次に受注数およびロット数のばらつきを表現するために、
受注数は平均 $\mu_{odder}=10$、分散 $\sigma^2_{odder}=2$ の正規分布に従う乱数から生成し、
ロット数は平均 $\mu_{lot}=3$、分散 $\sigma^2_{lot}=1$ の正規分布に従う乱数から生成する。
負の値および $0$ は現実的でないため除外し、正の値のみを採用する。

\begin{align}
    \mu_{odder} &\text{:総オーダー数の平均値} \nonumber\\
    \sigma^2_{odder} &\text{: 総オーダー数の分散} \nonumber\\
    \mu_{lot} &\text{: 各オーダーに注文された製品のロット数の平均値} \nonumber\\
    \sigma^2_{lot} &\text{: 各オーダーに注文された製品のロット数の分散} \nonumber\\
\end{align}

これにより、実際の生産現場に見られる受注量とロットサイズのゆらぎを簡易的に表現する。

各注文には、最大 $6$ 種類の製品種別が含まれ得るものとし、確率 $P=0.7$ で当該製品が注文に含まれると仮定する。
これにより,オーダーごとの製品構成行列 $odder$ を生成し、各製品種別ごとの総注文件数 $n_{s}$ はロット数の合計に基板1ロットあたり10枚という仮定を掛け合わせて算出する。
その後、各加工工程の処理時間は、この $n_{s}$ と機械別の単位加工時間または基準加工時間パラメータに基づいて決定する。
例えば、ドリル加工や電気検査など加工時間が枚数に比例すると考えられる工程では、$n_{s} \times MT_{m},m = \{1,2,3,4,5,8,10,11,12,13\}$ で工程時間を設定し、
一方、エッチングや積層などバッチ処理的な工程では、製品枚数に依存しない基準時間 $MT_{m},m = \{6,7,9\}$ を用いる。
こうして得られた処理時間行列が、行列 $OT$ に相当する。

\[
OT_{s,q}=
\begin{cases}
    n_{s} * MT_{m}  & m = \{1,2,3,4,5,8,10,11,12,13\} \\
    MT_{m} & m = \{6,7,9\}
\end{cases}
\]

以上のように、本研究の自動生成データは、
\begin{enumerate}
    \item 多層プリント基板の代表的な加工工程列を反映した機械割当行列 $OM$
    \item 受注数・ロット構成のばらつきと機械特性を考慮して決定される処理時間行列 $OT$
    \end{enumerate}
から構成され、実工場の特徴を保ちつつ、パラメータを変えることで多様な実験用ジョブセットを大量に生成できるように設計している。

\subsection{実際の生成手続き}
具体的なデータ生成手続きは、Python により実装したプログラムに従って以下の手順で行う。
\begin{enumerate}
  \item 乱数seed $seed_{value}=3$ を設定し、受注数およびロット数の正規分布パラメータ,機械別単位加工時間・基準加工時間パラメータを初期化する。
  \item 生成するジョブセット数を $gene=10000$ とし、$gene =0,\ldots, gene-1$ について以下の処理を繰り返す。
  \item 受注数の候補として、平均 $\mu_{odder}$、分散$\sigma^2_{odder}$ の正規分布からサンプルサイズ $odder_{size}$ 個の乱数系列を生成し、$0$ 以下の値を除外する。その中から1つをランダムに選択し、実際の受注数 $odder_{number}$ とする。
  \item 同様に,ロット数についても平均 $\mu_{lot}$,分散$\sigma^2_{lot}$ の正規分布からサンプルサイズ $lot_{size}$ 個の乱数系列を生成し、$0$ 以下を除外した上でロット数候補集合を得る。
  \item 受注数 $odder_{number}$ と製品種類数 $6$ に基づき、オーダー原始データ行列 $odder$(サイズ $odder_{number} \times count_{type}$)を初期化する。
        各注文 $O_{i}$ と製品種別 $J_{s}$ について、確率 $P=0.7$ で当該製品が注文に含まれるとし、含まれる場合にはロット数候補集合から1つをランダムに選択して $odder_{i,s}$ に代入する。
  \item 行列 $odder$ の列方向の合計に基板1ロットあたり10枚を乗じることで、各製品種別ごとの総基板枚数ベクトル $product_{count}$ を算出する。
  \item 各製品種別 $J_{s}$ および各工程番号 $q$ について、機械割当行列 $OM$ の要素 $OM_{s,q}$ を参照し、その値に応じて処理時間行列 $OT_{s,q}$ を決定する。
        枚数に比例する工程では $product_{count}[s]$ と単位時間を乗じ、バッチ処理工程では基準時間をそのまま用いる。
  \item 行列 $OM$ および $OT$ をもとに、行列 $JOBSET$ を作成する。
        $JOBSET$ は行方向に製品種別(製品1〜製品6)、列方向に工程番号(加工工程1〜加工工程17)を取り、各要素には「(機械番号, 加工時間)」の組を文字列として格納する。
        これにより、$JOBSET$ の機械番号部分が $OM$、加工時間部分が $OT$ に対応する。
  \item 生成条件(分布種別、受注数分布パラメータ、ロット数分布パラメータ、使用機械数,シード値)に基づいて出力ディレクトリおよびファイル名を構成し、$JOBSET$ を .csv 形式で保存する。
\end{enumerate}

    そして同じジョブの作業について、前作業と後作業があり、作業の前作業と後作業のどれかの作業機械は一致しないと設定する。

\begin{cases}
    before\{m_{s,q}\} \neq m_{s,q} \\
    after\{m_{s,q}\} \neq m_{s,q}
\end{cases}

\section{提案手法}
本研究では、JSSP問題に対して、GAを用いた探索手法を提案する。また、こういう加工順序が厳格に定められている問題には、PPS(Precedence Preservative Crossover)という交差手法を採用します。
\subsection{PPS交差法}
PPS交差法は、JSSP問題などの順序制約を伴う組合せ最適化問題において有効な交差手法の一つである。本手法は、親個体の遺伝子情報を組み合わせる際に、工程間の優先順序を保持しつつ、新たな子個体を生成することを目的としている。
\begin{enumerate}
    \item 2つの親染色体(Parent1, Parent2)を選択する。
    \item Parent1から順序情報を、Parent2から操作集合の構成を主に利用する。
    \item 子個体の染色体を初期化する。
    \item Parent1の順序に従い、Parent2に存在する操作を逐次選びながら、子個体に挿入する。ただし、すでに挿入された操作との順序制約を満たすように注意深く挿入される。
    \item すべての操作が挿入されるまで繰り返す。
    \end{enumerate}
数式:
\begin{itemize}
    \item $S = \{S_1, S_2, \dots, S_s\}$:製品種類の集合($s$は製品種類数)
    \item 各製品$S_s$は$q$個の加工操作を持ち、$f_s = \{f_{s1}, f_{s2}, \dots, f_{sq}\}$とする。
    \item 親個体は操作列として表現される:$P_1 = [o_1^1, o_2^1, \dots, o_m^1]$, $P_2 = [o_1^2, o_2^2, \dots, o_m^2]$
    \item 子個体$C$を空列として初期化:$C \leftarrow [\ ]$
    \end{itemize}

\begin{enumerate}
    \item $i \leftarrow 1$
    \item \textbf{while} $|C| < m$ \textbf{do}:
    \begin{enumerate}
        \item $o \leftarrow P_1[i]$ (親1の$i$番目の操作)
        \item \textbf{if} $o \in P_2$ かつ $\text{precedence\_satisfied}(o, C) = \text{True}$ \textbf{then}:
        \begin{itemize}
        \item $C \leftarrow C \cup \{o\}$ (子供に操作を追加)
        \item $P_2 \leftarrow P_2 \setminus \{o\}$
        \end{itemize}
        \item $i \leftarrow i + 1$
    \end{enumerate}
    \item \textbf{return} $C$
    \end{enumerate}

ここで、precedence\_satisfied関数は以下で定義される:

\[
\text{precedence\_satisfied}(f_{sq}, C) =
\begin{cases}
    \text{True} & \text{if } f_{s,(q-1)} \in C \text{ または } q = 1 \\
    \text{False} & \text{otherwise}
\end{cases}
\]

\clearpage

\begin{thebibliography}{99}

\bibitem{ref1} K.T. Chung, C.K.M.Lee and Y.P. Tsang, "Neural combinatorial optimization with reinforcement learning in industrial engineering: a survey"
\textit{Artif Intell Rev}. vol. 58, 130(2025).

\bibitem{ref2} M. Xu, Y. Mei, F.F. Zhang and M.J. Zhang, "Learn to optimise for job shop scheduling: a survey with comparison between genetic programming and reinforcement learning"
\textit{Artif Intell Rev}. vol. 58, 160(2025).

\bibitem{ref3} C. Ngwu, Y. Liu and R. Wu, "Reinforcement learning in dynamic job shop scheduling: a comprehensive review of AI-driven approaches in modern manufacturing"
\textit{J Intell Manuf}.(2025).

\bibitem{ref4} X.R. Shao and C.S. Kim, "An Adaptive Job Shop Scheduler Using Multilevel Convolutional Neural Network and Iterative Local Search"
\textit{IEEE Access}. vol. 10, pp. 88079-88092(2022).

\bibitem{ref5} L.B. Wang, X. Hu, Y. Wang, S. Xu, S. Ma, K. Yang, Z. Liu and W. Wang, "Dynamic job-shop scheduling in smart manufacturing using deep reinforcement learning"
\textit{Computer Networks}. vol. 190, 107937(2021).

\bibitem{ref6} S. Lee, Y. Cho and Y.H. Lee, "Injection Mold Production Sustainable Scheduling Using Deep Reinforcement Learning"
\textit{Sustainability}. vol. 12, 8718(2020).

\bibitem{ref7} C. Pickardt, J. Branke, T. Hildebrandt, J. Heger and B. Scholz-Reiter, "Generating dispatching rules for semiconductor manufacturing to minimize weighted tardiness"
\textit{Proceedings of the 2010 Winter Simulation Conference}.Baltimore, MD, USA, pp. 2504-2515(2010).

\bibitem{ref8} 村山 昇,川田 誠一, "遺伝的アルゴリズムを用いた加工機械と複積載AGVの同時スケジューリング"
\textit{日本機械学会論文集(C編)}. vol. 12, 712, pp. 3638-3643(2005-12).

\bibitem{ref9} B. Yu, Z. Hui, Z. Xin, C. Zheng, J. Riku, Y. Kun, "Toward Autonomous Multi-UAV Wireless Network: A Survey of Reinforcement Learning-Based Approaches"
\textit{IEEE COMMUNICATIONS SURVEYS AND TUTORIALS}. vol. 25, No. 4, FOURTH QUARTER 2023, pp. 3038-3067(2023).

\bibitem{ref10} 小野 啓介, 森川 克己, 長沢 敬祐, 高橋 勝彦, "フレキシブルジョブショップ環境の受託製造企業におけるエネルギー消費量配分問題"
\textit{J Jpn Ind Manage Assoc}. vol. 72, pp. 179-187(2022).

\bibitem{ref11} 貝原 俊也, 國領 大介, 藤井 信忠, 村上 亘, 梅田豊裕, "フレキシブルジョブショップを対象とした受注生産における機械稼働計画立案もための基礎検討"
\textit{第63回自動制御連合講演会}. (2020).

\bibitem{ref12} 貝原 俊也, 國領 大介, 藤井 信忠, 西村 翔平, "CPS型ファクトリセキュリティ実現に向けた生産スケジューリング手法に関する研究-マスカスタム生産対応フレキシブルオープンショップを対象とした検討-"
\textit{第61回自動制御連合講演会}. (2018).

\bibitem{ref13} 平沼 智之, 安田 翔也, 藤堂 健世, 谷口 茉帆, 山村 雅幸, "組合せ最適化ぬおけるベイジアン最適化アルゴリズムを組み込んだ遺伝的アルゴリズムの提案"
\textit{The 35th Annual Conference of the Japanese Society for Artificial Intelligence}. (2021).

\bibitem{ref14} 三神 賢雅, 伊原 滉也, 佐久間 拓人 , 加藤 昇平, "混合整数最適化に基づく生産ライン作業スケジュール生成システムの開発"
\textit{The 36th Annual Conference of the Japanese Society for Artificial Intelligence}. (2022).

\bibitem{ref15} 蘭 嘉, 田中 瑛理, 佐々木 優 , 森江 翔, 有馬 澄佳, "複数種のリソースを共用する多品種生産システムの分散協調スケジューリング-自動車部品後補充生産への適用-"
\textit{日本経営工学会論文誌}. vol. 72, No.1, pp. 75-87(2021).

\bibitem{ref16} G.Y. Shi, H. IIMA, N. Sannomiya, "A New Encodnig Scheme for Solving Job Shop Problems by Genetic Algorithm"
\textit{Conference on Decision and Control}. (1996).

\bibitem{ref17} UMIT BILGE, GUNDUZ ULUSOY, "A TIME WINDOW APPROACH TO SIMULTANEOUS SCHEDULING OF MACHINES AND MATERIAL HANDLING SYSTEM IN AN FMS"
\textit{Operations Research}.vol. 43, No.6, (1995).

\bibitem{ref18} 花田 良子, 廣安 知之, 三木 光範, "遺伝的アルゴリズムによる工場の生産スケジュールの自動生成"
\textit{THE SCIENCE AND ENGINEERING REVIEW OF DOSHISHA UNIVERSITY}.vol. 48, No.4, pp.241-248, (2008).

\bibitem{ref19} 平中 雄一朗, 西 竜志, 乾口 雅弘, "ラグランジュ緩和とカット生成による生産工程と複数台搬送車の同時スケジューリング問題に対する分解法"
\textit{システム制御情報学会論文誌}.vol. 20, No.12, pp.465-474, (2007).

\bibitem{ref20} Raghda B. Taha, Amin K. El-Kharbotly, Yomna M. Sadek, Nahid H. Afia, "A Genetic Algorithm for solving two-sided assembly line balancing problems"
\textit{Ain Shams Engineering Journal}.vol. 2, pp.227-240, (2011).

\bibitem{ref21} Kazi Shah Nawaz Ripon, N. H. Siddique, Jim Torresen, "Improved precedence preservation crossover for multi-objective job shop scheduling problem"
\textit{Evolving Systems}.vol. 2, pp.119-129, (2011).

\bibitem{ref22} Andrzej Kiernich, Jerzy Kalenik, Wojciech Steplewski, Marek Koscielski and Aneta Chołaj, "Impact of Particular Stages of the Manufacturing Process on the Reliability of Flexible Printed Circuits"
\textit{Sensors}.25, 140, (2025).

\bibitem{book1} 斎藤 康毅, 『ゼロから作るDeep Learning - Pythonで学ぶディープラーニングの理論と実装』,オライリー・ジャパン,2022.

\bibitem{book2} 野寺 隆志, 『楽々LATEX』,共立出版株式会社,1996.

\bibitem{book3} 飯塚 修平, 『ウェブ最適化ではじめる機械学習』,オライリー・ジャパン,2020.

\bibitem{book4} 矢沢 久雄, 『基本情報技術者 らくらく突破 Python』,技術評論社,2021.

\bibitem{book5} 伊藤 多一,今津 儀充,須藤 広大,など 『現場で使える! Python 深層強化学習入門 - 強化学習と深層学習による探索と制御』,株式会社翔泳社,2025.

\bibitem{book6} Kirill Bobrov, 『なっとく!並列処理プログラミング』,株式会社翔泳社,2025.

\bibitem{book7} 平井 有三, 『はじめてのパターン認識』,森北出版株式会社,2023.

\bibitem{book8} 大用 倉智,山田 孝子,『作りながら丁寧に学ぶ Pythonプログラミング入門』,関西学院大学出版会,2022.



\end{thebibliography}

\end{document}