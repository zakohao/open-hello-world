%\documentclass[twocolumn]{jarticle}
\documentclass{jarticle}

\usepackage{jsaiac}
\usepackage{graphicx}
\usepackage{amsmath}
\allowdisplaybreaks[4] 


%%
\title{%
\jtitle{遺伝的アルゴリズムと深層強化学習に基づく\\ジョブスケジュール最適化問題\\―PCB加工プロセスをモデルとして―}%
\etitle{Job Scheduling Optimization Based on Genetic Algorithms and Deep Reinforcement Learning\\―A Model of the PCB Manufacturing Process―}%
}
%%�p���͈ȉ����g�p
%\title{Style file for manuscripts of JSAI 20XX}

\jaddress{関西学院大学総合政策研究科, 兵庫県三田市学園上ヶ原, ili53007@kwansei.ac.jp}

\author{%
   \jname{汪 永豪\first{}}
   \ename{Wang Yonghao}
\and
   \jname{山田 孝子\second{}}
   \ename{Takako Yamada}
%\and
%Given-name Surname\third{}%%�p���͍����g�p
}

\affiliate{
\jname{\first{}関西学院大学大学院総合政策研究科}%
\ename{Graduate School of Policy Studies, Kwansei Gakuin University}%
\and
\jname{\second{}関西学院大学総合政策学部メディア情報学科}
\ename{Department of Applied Informatics, School of Policy Studies, Kwansei Gakuin University}
%\and
%\third{}Affiliation \#3 in English%%�p���͍����g�p
}

%%
%\Vol{28}        %% <-- 28th�i�ύX���Ȃ��ł��������j
%\session{0A0-00}%% <-- �u��ID�i�K�{)

\begin{abstract}
    With the globalization of the world economy, manufacturing companies must accept orders at lower prices than their competitors.
As a result, achieving low-cost and high-quality production management has become increasingly important, posing a challenging issue that all manufacturers must overcome.
To reduce costs, many companies have adopted Automated Guided Vehicles (AGVs) instead of human workers to transport parts between machines in a flexible manufacturing system (FMS).
Additionally, since machines in an FMS are not always in operation, it is crucial to utilize their idle time effectively by assigning them varying tasks.
This study aims to achieve simultaneous scheduling of both machines and AGVs, minimizing the completion time of existing tasks while making effective use of machine idle time to process additional parts. 
\end{abstract}

%\setcounter{page}{1}
\def\Style{``jsaiac.sty''}
\def\BibTeX{{\rm B\kern-.05em{\sc i\kern-.025em b}\kern-.08em%
 T\kern-.1667em\lower.7ex\hbox{E}\kern-.125emX}}
\def\JBibTeX{\leavevmode\lower .6ex\hbox{J}\kern-0.15em\BibTeX}
\def\LaTeXe{\LaTeX\kern.15em2$_{\textstyle\varepsilon}$}

\begin{document}
\maketitle

\section{はじめに}

世界経済のグローバル化に伴い、製造業企業は他社より安い価格で注文を受ける必要がある.
いかにして低コストな生産システムと高品質な品質管理を両立させ,安定的に維持するかは、製造業が必ず直面すると言って過言ではない難しい問題である.
昨今ではこの両立にさらに安全管理を含むワークライフバランスを考えた労務管理が加わり,生産現場では達成するべき課題が複雑化している.
コスト削減のため、多くな企業は人間作業員の代わりに、製造工程のロボットを採用し,工場内物流には複積載搬送ロボット(AGV)の導入をすすめている.
AGVの導入により生産機械間の半製品や仕掛品の運搬は自動化された.
生産計画には,スケジューリングだけではなく,AGVの搬送計画も含まれ,経路や運搬すべき半製品の決定までを生産計画に含むことになった.
本研究では,以下のシステムを順次作成し,生産管理計画システムとして実装し,最適解を求める計算機実験を実施,システムの性能を検証した.
%世界経済のグローバル化に伴い、製造業企業は他社より安い価格で注文を受ける必要がある.
%いかにして低コストと高品質な生産管理がより重要になり、製造業であれば誰でも乗り越えなければならない難しい問題である.
%コスト削減のために、多くな企業が人間作業員の代わりに、複積載搬送ロボット(AGV)を採用し、フレキシブルシステム生産で、部品を機械の間に運ぶ方法を採用しました.
%また、FMS生産には機械はいつも働いているわけではないので、その機械の待つ時間を活かせるために、ばらつきな仕事を機械にやらせることにしよう.
%本研究では機械とAGV両方同時スケジューリングを求める上に、現有な仕事を出来るだけ最短時間で完成させ、さらに、本来機械が遊んでいる時間を生かして、部品を加工するシステムを開発した.
%%
本研究でこれまで行ったことをまとめる
\begin{enumerate}
    \item 実験用ジョブセットとして,製品種別ごとの加工機械の利用順番,利用時間の組み合わせデータの自動生成システムの作成
    \item 生成されたデータを用いた生産機械のスケジューリングを総当たり法で厳密解を求める最短生産スケジュール算出システムの作成
    \item 生産スケジュールの可視化システムの作成
    \item 遺伝的アルゴリズムをベースとする,緩和手法による許容解算出と解の改善システムの作成
    \item 生産機械に加え複数台AGVを導入した生産スケジュールの最適化システムの作成
    \item 求められた生産スケジュールに基づく生産機械の遊休時間活用する余剰生産品提案システムの作成
    \end{enumerate}
    以上を行ったうえで、本研究では機械とAGV両方同時スケジューリングを求め、現有な仕事を最短時間で終了する許容解を導出し,その許容解のもとで,機械の遊休時間を活用する余剰製品を提案するシステムの開発に取り組む.
    このような問題は、JSSP(Job Shop Scheduling Problem)問題と言います。

%\section{符号定義}
%\begin{align}
    %O &\text{: 工場が受領した注文書の集合} \nonumber\\
    %O_{i} &\text{: 第$i$番に工場が受領した注文書} \nonumber\\
    %OT &\text{: すべての注文書に対してもっとも遅い納期} \nonumber\\
    %OT_{i} &\text{: 第$i$番に工場が受領した注文書の納期} \nonumber\\
    %O_{i}S_{s} &\text{: 第$i$注文書に記載された注文製品$S$の種別番号$s$} \nonumber\\
    %K_{is} &\text{: 第$i$注文書で受注した製品$S$の種別番号$s$の価格} \nonumber\\
    %n_{is} &\text{: 第$i$注文書に記載された第$s$種別番号製品の注文個数} \nonumber\\
    %n_{s} &\text{: すべての注文書に記載された第$s$種別番号製品の合計注文個数} \nonumber\\
    %n_{s} &= \sum_{i=1}^{\infty} n_{is} \nonumber\\
    %N &\text{: すべての製品の合計注文個数} \nonumber\\
    %N &= \sum_{s=1}^{\infty} n_{s} \nonumber\\
    %S_{s}Q_{q} &\text{: 第$s$種別製品の第$q$番目の加工工程} \nonumber\\
    %m_{sq} &\text{: 第$s$種別製品の第$q$番目の加工工程に割り当てられる機械番号} \nonumber\\
    %M &\text{: 使用可能な加工機械の集合} \nonumber\\
    %m_{m} &\text{: 第$m$番の加工機械} \nonumber\\
    %t_{sq} &\text{: 第$s$種別製品の第$q$番目の加工工程にかかる加工時間} \nonumber\\
    %ht_{l_{1}l_{2}} &\text{: AGVが$l_{1}$から$l_{2}$までの搬送時間} \nonumber\\
    %l_{1},l_{2} &\subseteq (L/U,m_{1},\cdots,m_{m})  \nonumber
%\end{align}

\section{先行研究}
グローバル経済の進展や企業のコスト削減要求に伴い、生産現場では柔軟で効率的な生産体制の構築が求められている。
その中で、需要の多様化に対応するために、少量多品種生産が一般的となり、このような生産形態に適したスケジューリングの最適化が重要な課題となっている。
特に、ジョブショップスケジューリング問題(Job Shop Scheduling Problem: JSSP)は、生産システムにおける代表的な最適化課題として広く研究されている。
JSSPは、複数のジョブを限られた機械に割り当て、全体の処理時間や納期遅延を最小化することを目的とする問題である。
また、各ジョブには処理順序が定められており、この順序を厳密に守る必要がある点が特徴である。
この問題はNP困難であり、最適解を求めることは計算量的に非常に困難である。
そのため、これまで多くの研究では、遺伝的アルゴリズム(Genetic Algorithm: GA)や焼きなまし法(Simulated Annealing: SA)などのヒューリスティック手法を用いて、効率的に近似解を求めるアプローチが提案されてきた。
しかし、実際の製造現場では、機械の故障や新規ジョブの追加、処理時間の変動など、動的な要素が頻繁に発生する。そのような動的環境下では、あらかじめ定められたルールに基づくヒューリスティック手法では十分に対応できない場合がある。この問題を解決するため、近年では強化学習(Reinforcement Learning: RL)を用いて、環境変化に応じて自律的にスケジューリング方策を学習する手法が注目されている。
静的な環境においてはGAやSAなどの伝統的アルゴリズムが一定の成果を上げている一方で、動的な生産環境では強化学習に基づくアプローチのほうが、より高い適応性と柔軟性を示すことが報告されている。このように、JSSPにおけるスケジューリング最適化の研究は、従来の探索型手法から学習型手法へと発展しつつある。

K.T. Chungら\cite{ref1}は現在工業5.0の時代に向けて、機械学習、特に強化学習を用いて、CO(combinatorial optimization)問題を解決しようについて、現在の研究進展を述べた。
M. Xuら\cite{ref2}は遺伝的プログラミングと強化学習を統合的に整理・比較しはJSSPスケジュール最適化問題ににおいて果たす役割と近年の研究動向を明らかにした。
C. Ngwuら\cite{ref3}は動的なJSSP問題は強化学習や進化的ヒューリスティック、機械学習を用いた手法でリアルタイムな意思決定は可能であることを示しました。
X.R. Shaoら\cite{ref4}は小規模かつ短時間のJSSP問題に対して、多層畳み込みニューラルネットワーク(ML-CNN)と反復局所探索(ILS)を組み合わせ、ML-CNNで全体経路を学習し、ILSで最適な局所経路を探索する手法が提案した。
L.B. Wangら\cite{ref5}は機械故障や再作業などの動的な環境でのJSSP問題につい深層強化学習(DRL)を用いた動的スケジューリング手法が提案され、PPOにより最適な方策を学習することで、リアルタイムな生産スケジューリングが可能であることが示された。
S. Leeら\cite{ref6}は射出成形金型製造プロセスにおいて、Deep Q-Networkを活用することで従来より総加重遅延を低減できる金型生産スケジューリングを最適化した。
C. Pickardtら\cite{ref7}は半導体製造プロセスにおいて、遺伝的プログラミング(GP)と確率的離散事象シミュレーションを組み合わせることで自動的に従来のルールより性能優れたなルールを生成した。
村山ら\cite{ref8}はJSSP問題に基づき、加工機械と複数積載型AGVの同時スケジューリングを対象とした遺伝的アルゴリズム(GA)手法が提案した。

\section{モデル}
\subsection{モデル仮定}
本モデルでは受注生産を行う工場の製造工程をモデル化する.
\begin{enumerate}
    \item 注文票はさまざまな注文主からくる。
    \item 各注文書には製品$S$の種別が1種類が記載されている。注文する製品種別が異なる製品は異なる注文書で発注される.
    \item 注文書ごとに価格、納期、ロット数がある。
    \item 工場は、出荷する出荷日(納期)が同じ注文書$O_{\textit{i}}$をまとめ、これを注文書セットとする。
    \item 製造する製品種別に応じて、使用する加工機械、加工時間とその加工機械の使用順が予め与えられる。
    \item 工場は注文書セットで同一出荷として受注した注文書を納期日までに出荷できるよう、加工機械とAGVの最適なスケジューリングを行う。
    \item スケジューリングの結果得られた注文書セットに含まれる受注した製品を定められた個数すべて出荷できる日を最速出荷日とする.また最速出荷日は早いほどよいとする.
    \item 機械は注文書に記載された製品種別で指定された注文個数をすべて加工が終了するまで,作業を中途で終えたり,途中で他の製造作業を行うような割り込みはできないものとする。
    \item 製品種別が異なるための機械のセットアップタイムは考慮しない。
    \item 機械は製品種別ごとに定まる順番の通りに機械を使用して加工作業を行わなければならない。
    \item 出荷が等しい注文書,すなわちある注文書セットがすべて出荷可能とればその注文書セットでの加工作業は終了となる。
    \end{enumerate}

\subsection{受注}
製品を製造する工場をここでは考える.一定の期間に工場には様々な顧客から注文票が到着する.ある期間中に到着した注文票を$0$とあらわす.この注文票の集合は$O_1$から$O_i$までの$i$枚の注文票からなる.各注文票$O_i$には,注文する製品名$S_s$,納期$d_i$,各製品名注文ロット数$lnS_{is}$,注文する製品の単価$price_s$が記載されている.注文票に記載できる製品は1種類で,複数の製品を発注する場合には,注文票を分けることになっている.
工場は当該期間に受注した注文票で最も納期の早い注文票の納期を$d_{min}$として,その期日に間に合うように,受注した製品の製造計画を立てる.
\subsection{製造}
製造計画の作成するためには,まず製品種別$S_s$ごとに,加工工程として使用する製造機械の順番が決まっている.製造機械を使用する時間は製造する個数による.そのため製品$S_{s}$を製造するための$q$番目の加工工程を$f_{sq}$ としてあらわし,そこで使われる製造機械番号を$m_{sq}$,加工時間を$t_{sq}$で表す.

\subsection{仕掛品搬送}
加工工程で、異なる製造機械の間で仕掛品搬送にAGV(Automated Guided Vehicle)を用いる場合,製品$S_s$の第$f_{sq}$工程と第$f_{s,(q+1)}$工程間を$AGV_{j}$で搬送するとき,その輸送時間を$ut_{j,s,q,(q+1)}$で表す.
以上を記号として整理すると,以下のようになる.

\section{符号定義}
\begin{align}
    O &\text{: 工場が受領した注文書の集合} \nonumber\\
    O_{i} &\text{: 第$i$番に工場が受領した注文書} \nonumber\\
    d_{min} &\text{: 受注した注文票で最も納期の早い注文票の納期} \nonumber\\
    d_{i} &\text{: 第$i$番に工場が受領した注文書の納期} \nonumber\\
    S{s} &\text{: 製品$S$の種別番号$s$} \nonumber\\
    S{is} &\text{: 第$i$注文書に記載された注文製品$S$の種別番号$s$} \nonumber\\
    price_{s} &\text{: 第$i$注文書で受注した製品$S$の種別番号$s$の単価} \nonumber\\
    lnS_{is} &\text{: 第$i$注文書に記載された第$s$種別番号製品の注文ロット数} \nonumber\\
    n_{s} &\text{: すべての注文書に記載された第$s$種別番号製品の合計注文個数} \nonumber\\
    n_{s} &= 10\sum_{i=1}^{\infty} lnS_{is} \nonumber\\
    N &\text{: すべての製品の合計注文個数} \nonumber\\
    N &= \sum_{s=1}^{\infty} n_{s} \nonumber\\
    f_{sq} &\text{: 第$s$種別製品の第$q$番目の加工工程} \nonumber\\
    m_{sq} &\text{: 第$s$種別製品の第$q$番目の加工工程に割り当てられる機械番号} \nonumber\\
    M &\text{: 使用可能な加工機械の集合} \nonumber\\
    m_{m} &\text{: 第$m$番目の加工機械} \nonumber\\
    t_{sq} &\text{: 第$s$種別製品の第$q$番目の加工工程にかかる加工時間} \nonumber\\
    l_{1},l_{2} &\subseteq (L/U,m_{1},\cdots,m_{m})  \nonumber\\
    l_{1}& \ne l_{2} \nonumber\\
    AGV_{j} & \text{: 第$j$番目のAGV} \nonumber\\
    ht_{j,l_{1},l_{2}} &\text{: $AGV_{j}$が$l_{1}$から$l_{2}$までの搬送時間} \nonumber\\
    ut_{j,s,q,(q+1)} &\text{: $AGV_{j}$が製品$s$の第$q$工程と第$q+1$工程間の輸送時間} \nonumber
\end{align}

\subsection{実験用ジョブセット生成}
一般的に企業への注文書の具体例は外部には公開されない.そのため本研究では注文書セットを予め様々な確率的に生成する計算機実験用の自動生成システムを作成し,実験用の注文書セットを作成する.
本節ではこの自動生成の手続きとそれにより作成した注文書セットについて述べる.
\subsection{記号定義}
\begin{align}
    \mu_{odder} &\text{: 納期内の総オーダー数の平均値} \nonumber\\
    %\mu_{odder} &\text{= 10} \nonumber\\
    \sigma^2_{odder} &\text{: 納期内の総オーダー数の分散} \nonumber\\
    %\sigma^2_{odder} &\text{= 2} \nonumber\\
    \mu_{lot} &\text{: 各オーダーに注文された製品のロット数の平均値} \nonumber\\
    %\mu_{lot} &\text{= 3} \nonumber\\
    \sigma^2_{lot} &\text{: 各オーダーに注文された製品のロット数の分散} \nonumber\\
    %\sigma^2_{lot} &\text{= 1} \nonumber\\
\end{align}

\subsection{生成ルール}
\begin{enumerate}
    \item 多層回路基盤を作る工場等、同じ工程が何回も加工する加工工程に向けて、生成する。
    \item 正規分布を用いてオーダー数$O_{i}$を生成($\mu_{odder}$ = 10、$\sigma^2_{odder}$ = 2)。
    \item 負の値やゼロを除外し、サンプルからランダムに1つの整数を選んで、今回のシミュレーションの注文数とする。
    \item 各注文は最大で6種類の製品を含む。
    \item 各種別の製品について、固定されている制作工程がある、工程間は転倒しない。
    \item 各製品のロット数は正規分布($\mu_{lot}$ = 3、$\sigma^2_{lot}$ = 1)から生成し、負またはゼロを除外。
    \item 各製品は注文内で0.7の確率で含まれるによる制御。
    \item 各製品の総数量は、すべての注文におけるその製品のロット数合計に10を掛けて算出。
    \item $m_{1}$から$m_{13}$まで13個それぞれ役目が重ねない加工機械がある。
    \item 各加工工程の操作時間 $T_{J}$:一部は製品数量に単位時間を掛けて決定、一部は固定時間。
    \item 各製品種別に対し、加工工程と対応する操作時間の組み合わせを記録する。
    \end{enumerate}

    同じジョブの作業について、前作業と後作業があり、作業の前作業と後作業のどれかの作業機械は一致しない.\\

\begin{enumerate}
    \item $before\{m_{sq}\} \neq m_{sq}$ \\
    \item $after\{m_{sq}\} \neq m_{sq}$ \\
    \end{enumerate}

\subsection{レイアウト、AGV搬送設定}
\begin{enumerate}
    \item 工場にはL/U(Lording/Unlording)、機械($m_{1} \sim m_{m}$)がある。
    \item 各AGVはL/Uから出発、各機械の間を部品を搬送する。
    \item 製品の最後の一つ加工操作が終わった後に、AGVは最後の一つの加工操作を務めた機械のところで、部品を受取、L/Uまで搬送する。
    \item 注文書に書かれているすべての製品がL/Uまで搬送したら、加工終了し総操作時間加算する。
    \item AGVは一回限りに一種類の部品を搬送します、容量は考慮しません。
    \item $AGV_j$がL/Uから各機械への搬送時間は同じく$ht_{j,L/U,m_{m}}=30$。
    \item AGVが機械の間での搬送時間は、相隣の機械は同じく20、機械の間に挟まれた機械の個数分かける10加算する。
    \item 一つの搬送作業に対して、AGV1台で搬送する。
    \item 一つの搬送作業に対して、一回しかしない。
    \end{enumerate}

\section{提案手法}
本研究では、JSSP問題に対して、GAを用いた探索手法を提案する。また、こういう加工順序が厳格に定められている問題には、PPS(Precedence Preservative Crossover)という交差手法を採用します。
\subsection{PPS交差法}
PPS交差法は、JSSP問題などの順序制約を伴う組合せ最適化問題において有効な交差手法の一つである。本手法は、親個体の遺伝子情報を組み合わせる際に、工程間の優先順序を保持しつつ、新たな子個体を生成することを目的としている。
\begin{enumerate}
    \item 2つの親染色体(Parent1, Parent2)を選択する。
    \item Parent1から順序情報を、Parent2から操作集合の構成を主に利用する。
    \item 子個体の染色体を初期化する。
    \item Parent1の順序に従い、Parent2に存在する操作を逐次選びながら、子個体に挿入する。ただし、すでに挿入された操作との順序制約を満たすように注意深く挿入される。
    \item すべての操作が挿入されるまで繰り返す。
    \end{enumerate}
数式:
\begin{itemize}
    \item $S = \{S_1, S_2, \dots, S_s\}$:製品種類の集合($s$は製品種類数)
    \item 各製品$S_s$は$q$個の加工操作を持ち、$f_s = \{f_{s1}, f_{s2}, \dots, f_{sq}\}$とする。
    \item 親個体は操作列として表現される:$P_1 = [o_1^1, o_2^1, \dots, o_m^1]$, $P_2 = [o_1^2, o_2^2, \dots, o_m^2]$
    \item 子個体$C$を空列として初期化:$C \leftarrow [\ ]$
    \end{itemize}

\begin{enumerate}
    \item $i \leftarrow 1$
    \item \textbf{while} $|C| < m$ \textbf{do}:
    \begin{enumerate}
        \item $o \leftarrow P_1[i]$ (親1の$i$番目の操作)
        \item \textbf{if} $o \in P_2$ かつ $\text{precedence\_satisfied}(o, C) = \text{True}$ \textbf{then}:
        \begin{itemize}
        \item $C \leftarrow C \cup \{o\}$ (子供に操作を追加)
        \item $P_2 \leftarrow P_2 \setminus \{o\}$
        \end{itemize}
        \item $i \leftarrow i + 1$
    \end{enumerate}
    \item \textbf{return} $C$
    \end{enumerate}

ここで、precedence\_satisfied関数は以下で定義される:

\[
\text{precedence\_satisfied}(f_{sq}, C) =
\begin{cases}
    \text{True} & \text{if } f_{s,(q-1)} \in C \text{ または } q = 1 \\
    \text{False} & \text{otherwise}
\end{cases}
\]
    
\begin{thebibliography}{99}

\bibitem{ref1} K.T. Chung, C.K.M.Lee and Y.P. Tsang, "Neural combinatorial optimization with reinforcement learning in industrial engineering: a survey"
\textit{Artif Intell Rev}. vol. 58, 130(2025).

\bibitem{ref2} M. Xu, Y. Mei, F.F. Zhang and M.J. Zhang, "Learn to optimise for job shop scheduling: a survey with comparison between genetic programming and reinforcement learning"
\textit{Artif Intell Rev}. vol. 58, 160(2025).

\bibitem{ref3} C. Ngwu, Y. Liu and R. Wu, "Reinforcement learning in dynamic job shop scheduling: a comprehensive review of AI-driven approaches in modern manufacturing"
\textit{J Intell Manuf}.(2025).

\bibitem{ref4} X.R. Shao and C.S. Kim, "An Adaptive Job Shop Scheduler Using Multilevel Convolutional Neural Network and Iterative Local Search"
\textit{IEEE Access}. vol. 10, pp. 88079-88092(2022).

\bibitem{ref5} L.B. Wang, X. Hu, Y. Wang, S. Xu, S. Ma, K. Yang, Z. Liu and W. Wang, "Dynamic job-shop scheduling in smart manufacturing using deep reinforcement learning"
\textit{Computer Networks}. vol. 190, 107937(2021).

\bibitem{ref6} S. Lee, Y. Cho and Y.H. Lee, "Injection Mold Production Sustainable Scheduling Using Deep Reinforcement Learning"
\textit{Sustainability}. vol. 12, 8718(2020).

\bibitem{ref7} C. Pickardt, J. Branke, T. Hildebrandt, J. Heger and B. Scholz-Reiter, "Generating dispatching rules for semiconductor manufacturing to minimize weighted tardiness"
\textit{Proceedings of the 2010 Winter Simulation Conference}.Baltimore, MD, USA, pp. 2504-2515(2010).

\bibitem{ref8} 村山 昇,川田 誠一, "遺伝的アルゴリズムを用いた加工機械と複積載AGVの同時スケジューリング"
\textit{日本機械学会論文集(C編)}. vol. 12, 712, pp. 3638-3643(2005-12).


\end{thebibliography}

\end{document}