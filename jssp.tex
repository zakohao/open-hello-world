\documentclass[twocolumn]{jarticle}

\usepackage{jsaiac}
\usepackage{graphicx}
\usepackage{amsmath}


%%
\title{
\jtitle{FMSにおける加工機械と複積載搬送ロボットの\\同時スケジューリング問題}
\etitle{Simultaneous Scheduling Problem of Processing Machines and Multi-Load Transport Robots \\in the FMS}
}
%%�p���͈ȉ����g�p
%\title{Style file for manuscripts of JSAI 20XX}

\jaddress{関西学院大学総合政策研究科, 兵庫県三田市学園上ヶ原, ili53007@kwansei.ac.jp}

\author{%
   \jname{汪 永豪\first{}}
   \ename{Wang Yonghao}
\and
   \jname{山田 孝子\second{}}
   \ename{Takako Yamada}
%\and
%Given-name Surname\third{}%%�p���͍����g�p
}

\affiliate{
\jname{\first{}関西学院大学大学院総合政策研究科}%
\ename{Graduate School of Policy Studies, Kwansei Gakuin University}%
\and
\jname{\second{}関西学院大学総合政策学部メディア情報学科}
\ename{Department of Applied Informatics, School of Policy Studies, Kwansei Gakuin University}
%\and
%\third{}Affiliation \#3 in English%%�p���͍����g�p
}

%%
%\Vol{28}        %% <-- 28th�i�ύX���Ȃ��ł��������j
%\session{0A0-00}%% <-- �u��ID�i�K�{)

\begin{abstract}
    With the globalization of the world economy, manufacturing companies must accept orders at lower prices than their competitors.
As a result, achieving low-cost and high-quality production management has become increasingly important, posing a challenging issue that all manufacturers must overcome.
To reduce costs, many companies have adopted Automated Guided Vehicles (AGVs) instead of human workers to transport parts between machines in a flexible manufacturing system (FMS).
Additionally, since machines in an FMS are not always in operation, it is crucial to utilize their idle time effectively by assigning them varying tasks.
This study aims to achieve simultaneous scheduling of both machines and AGVs, minimizing the completion time of existing tasks while making effective use of machine idle time to process additional parts. 
\end{abstract}

%\setcounter{page}{1}
\def\Style{``jsaiac.sty''}
\def\BibTeX{{\rm B\kern-.05em{\sc i\kern-.025em b}\kern-.08em%
 T\kern-.1667em\lower.7ex\hbox{E}\kern-.125emX}}
\def\JBibTeX{\leavevmode\lower .6ex\hbox{J}\kern-0.15em\BibTeX}
\def\LaTeXe{\LaTeX\kern.15em2$_{\textstyle\varepsilon}$}

\begin{document}
\maketitle

\section{はじめに}

世界経済のグローバル化に伴い、製造業企業は他社より安い価格で注文を受ける必要がある.
いかにして低コストな生産システムと高品質な品質管理を両立させ,安定的に維持するかは、製造業が必ず直面すると言って過言ではない難しい問題である.
昨今ではこの両立にさらに安全管理を含むワークライフバランスを考えた労務管理が加わり,生産現場では達成するべき課題が複雑化している.
コスト削減のため、多くな企業は人間作業員の代わりに、製造工程のロボットを採用し,工場内物流には複積載搬送ロボット(AGV)の導入をすすめている.
AGVの導入により生産機械間の半製品や仕掛品の運搬は自動化された.
生産計画には,スケジューリングだけではなく,AGVの搬送計画も含まれ,経路や運搬すべき半製品の決定までを生産計画に含むことになった.
本研究では,以下のシステムを順次作成し,生産管理計画システムとして実装し,最適解を求める計算機実験を実施,システムの性能を検証した.
%世界経済のグローバル化に伴い、製造業企業は他社より安い価格で注文を受ける必要がある.
%いかにして低コストと高品質な生産管理がより重要になり、製造業であれば誰でも乗り越えなければならない難しい問題である.
%コスト削減のために、多くな企業が人間作業員の代わりに、複積載搬送ロボット(AGV)を採用し、フレキシブルシステム生産で、部品を機械の間に運ぶ方法を採用しました.
%また、FMS生産には機械はいつも働いているわけではないので、その機械の待つ時間を活かせるために、ばらつきな仕事を機械にやらせることにしよう.
%本研究では機械とAGV両方同時スケジューリングを求める上に、現有な仕事を出来るだけ最短時間で完成させ、さらに、本来機械が遊んでいる時間を生かして、部品を加工するシステムを開発した.

\begin{enumerate}
    \item 実験用ジョブセットとして,製品種別ごとの加工機械の利用順番,利用時間の組み合わせデータの自動生成システムの作成
    \item 生成されたデータを用いた生産機械のスケジューリングを総当たり法で厳密解を求める最短生産スケジュール算出システムの作成
    \item 生産スケジュールの可視化システムの作成
    \item 遺伝的アルゴリズムをベースとする,緩和手法による許容解算出と解の改善システムの作成
    \item 生産機械に加え複数台AGVを導入した生産スケジュールの最適化システムの作成
    \item 求められた生産スケジュールに基づく生産機械の遊休時間活用する余剰生産品提案システムの作成
    \end{enumerate}
        本研究では機械とAGV両方同時スケジューリングを求め、現有な仕事を最短時間で終了する許容解を導出し,その許容解のもとで,
    機械の遊休時間を活用する余剰製品を提案するシステムの開発を行った.

\section{符号定義}

    $\textit{O}$ : すべての注文主からの注文の集合\\
    $O_{\textit{i}}$ : 注文主\textit{i}からの注文\\
    $\textit{O}_{\textit{i}} \subseteq \textit{O}$ , \textit{i} = \{1,2,3,\textellipsis\}\\
    $\textit{OT}$ : すべての注文の納期\\
    $OT_{\textit{i}}$ : 注文主\textit{i}からの注文の納期 , \textit{i} = \{1,2,3,\textellipsis\}\\
    $\textit{OT} \geq OT_{\textit{i}}$\\
    $O_{\textit{i}}S_{\textit{s}}$  : 注文主\textit{i}からの注文の第\textit{s}種製品がある , \textit{s} = \{1,2,3,\textellipsis\}\\
    $n_{\textit{is}}$ : 注文主\textit{i}からの注文の第\textit{s}種製品の要求個数\\
    $n_{\textit{s}}$ : すべての注文主からの注文の第\textit{s}種製品の要求個数\\
    $n_{\textit{s}} = \sum_{i=1}^{\infty} n_{\textit{is}}$ \\
    $\textit{N}$ : すべての製品の要求個数\\
    $\textit{N} = \sum_{s=1}^{\infty} n_{\textit{s}}$\\
    $S_{\textit{s}}Q_{\textit{q}}$  : 第\textit{s}種製品の第\textit{q}番目加工工程\\
    $m_{\textit{sq}}$ : 第\textit{s}種製品の第\textit{q}番目加工工程の加工機械番号\\
    $\textit{M}$ : 加工機械の集合\\
    $\textit{m}_{\textit{m}}$ : 第\textit{m}台加工機械\\
    $\textit{m}_{\textit{m}} \subseteq \textit{M}$ , \textit{m} = \{1,2,\textellipsis,9,10\}\\
    $t_{\textit{sq}}$ : 第\textit{s}種製品の第\textit{q}番目加工工程の加工時間\\



\section{モデル仮定}
\begin{enumerate}
    \item 注文はいろいろな注文主からくる。
    \item 注文書ごとに1種類以上の製品がある。
    \item 注文書ごとに製品種別\textit{s}ごとには納期、個数がある。
    \item 工場は、納期\textit{T}に出荷する注文$O_{\textit{i}}$をまとめる。
    \item 注文ごとの製品に応じて、使用する加工機械、加工時間と加工順番が予め設定される。
    \item 工場は一つの納期内に注文を満たして製品を出荷できるよう、加工機械とAGVの最適なスケジューリング立つ。
    \item 機械は作業途中で中退できない、割り込みできない。
    \item ゼットタイムは考慮しない。
    \item 加工工程はあらかじめに設定している順番通りに加工する。
    \item 第\textit{s}種製品すべての加工工程を終える次第、加工終了。
    \end{enumerate}

\section{実験用ジョブセット生成ルール設定}
    企業のジョブセットデータは企業機密の故に、一般的に公開されない.
本研究に使用しているジョブセットはすべて自動生成されたものだ.
次の章には、種類三つ、それぞれ10000個のジョブセット生成ルールと符号定義を紹介する.

    \begin{enumerate}
    \item \label{item:setting} 符号定義と制約条件
    
    %\textit{J} : ジョブの集合\\
    %\textit{P} : 作業の集合\\
    %\textit{M} : 加工機械の集合\\
    %$\textit{M}_{\textit{m}} \subseteq \textit{M}$\\
    %$t_{\textit{ij}}$ : ジョブi作業jの作業時間\\
    %$P_{\textit{ij}}$ : ジョブiの作業j\\
    %$\textit{M}(P_{\textit{ij}})$ : ジョブi作業jの作業機械番号\\
    %$\mu$ : 平均値\\
    %$\sigma$ : 標準偏差\\
    %$\sigma^2$ : 分散\\
    
    同じジョブの作業について、前作業と後作業があり、作業の前作業と後作業のどれかの作業機械は一致しない.\\

    $before\{\textit{M}(P_{\textit{ij}})\}$ $\neq$ \textit{M}$P_{\textit{i(j-1)}}$ \\
    $after\{\textit{M}(P_{\textit{ij}})\}$ $\neq$ \textit{M}$P_{\textit{i(j+1)}}$ \\

    \item 正規分布
    
    ジョブセットのジョブ数平均値 : $\mu_{\textit{job}}$ = 5\\
    ジョブセットのジョブ数分散 : $\sigma^2_{\textit{job}}$ = 1\\
    ジョブiの作業数平均値 : $\mu_\textit{job(i)}$ = 5\\
    ジョブiの作業数分散 : $\sigma^2_\textit{job(i)}$ = 1\\
    $\textit{M}_{\textit{m}} \subseteq \textit{M}$ , m = \{1,2,3,4\}\\
    ジョブセットの各操作時間平均値 : $\mu_{\textit{t}}$ = 15\\
    ジョブセットの各操作時間分散 : $\sigma^2_{\textit{t}}$ = 3\\

    \item 二谷正規分布
    
    一つ目のジョブセットのジョブ数平均値 : $\mu_{\textit{job1}}$ = 5\\
    一つ目のジョブセットのジョブ数分散 : $\sigma^2_{\textit{job1}}$ = 1\\
    二つ目のジョブセットのジョブ数平均値 : $\mu_{\textit{job2}}$ = 8\\    二つ目のジョブセットのジョブ数分散 : $\sigma^2_{\textit{job2}}$ = 1\\
    一つ目のジョブiの作業数平均値 : $\mu_\textit{job1(i)}$ = 5\\
    一つ目のジョブiの作業数分散 : $\sigma^2_\textit{job1(i)}$ = 1\\
    二つ目のジョブiの作業数平均値 : $\mu_\textit{job2(i)}$ = 5\\
    二つ目のジョブiの作業数分散 : $\sigma^2_\textit{job2(i)}$ = 1\\
    $\textit{M}_{\textit{m}} \subseteq \textit{M}$ , m = \{1,2,3,4\}\\
    一つ目のジョブセットの各操作時間平均値 : $\mu_{\textit{t1}}$ = 10\\
    一つ目のジョブセットの各操作時間分散 : $\sigma^2_{\textit{t1}}$ = 1\\
    二つ目のジョブセットの各操作時間平均値 : $\mu_{\textit{t2}}$ = 10\\
    二つ目のジョブセットの各操作時間分散 : $\sigma^2_{\textit{t2}}$ = 1\\

    \item 一様分布
    
    ジョブセットのジョブ数$\alpha_{\textit{job}}$ : $\alpha_{\textit{job}}$ =  \{3,4,\textellipsis,8,9\}\\
    ジョブiの作業数 $\alpha_{\textit{job(i)}}$ : $\alpha_{\textit{job(i)}}$ = \{3,4,\textellipsis,6,7\}\\
    $\textit{M}_{\textit{m}} \subseteq \textit{M}$ , m = \{1,2,3,4\}\\
    ジョブi作業jの作業時間 : $t_{\textit{ij}}$ = \{10,11,\textellipsis,18,19\}\\
    
    \end{enumerate}

\end{document}
